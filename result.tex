\section{実験}
侵入検知問題である10\%KDD'99データセットを使用して,
scikit-learnのデフォルトパラメータ,既存手法,提案手法の間で表~\ref{tab:1}の設定で比較実験を行った.
実験は,データセットからランダムに10\%抽出した学習セット,検証セット,テストセットの3つのデータセットを用意して行った.
検証セットはABCアルゴリズムにおける解の評価に使われ,
得られた最適解を未知のデータであるテストセットの分類に適用し
,その結果を最終的な分類精度とする.
\subsection{データセット}
10\%KDD'99データセットは,オリジナルのKDD'99データセットのうちサンプル数の多いnormal,
dos,probeクラスを10\%に減らしたデータセットである.また,KDD'99データセットの特徴数は41個である.
KDD'99データセットの内訳を表~\ref{kdd99}に示す.
\begin{table}[tb]
    \centering
    \begin{minipage}{0.45\textwidth}  % 表の幅を指定
        \centering
        \caption{KDD'99データセットの内訳}  % 表のキャプション
        \begin{tabular}{|l|r|r|}  % 左揃えと右揃えに変更
          \hline  % 横線
          クラス & KDD'99 & 10\%KDD'99 \\  % ヘッダー行
          \hline  % 横線
          normal & 972780 & 97279 \\  % 1行目
          \hline  % 横線
          dos & 3883370 & 391458 \\  % 2行目
          \hline  % 横線
          probe & 41102 & 4107 \\  % 3行目
          \hline  % 横線
          r2l & 1126 & 1126 \\  % 4行目
          \hline  % 横線
          u2r & 52 & 52 \\  % 5行目
          \hline  % 横線
          合計 & 4898430 & 494021 \\  % 合計行
          \hline  % 横線
        \end{tabular}
        \label{kdd99}  % 表のラベル 
    \end{minipage}
    \begin{minipage}{0.45\textwidth}  % 表の幅を指定
        \centering
        \caption{実験データセットの内訳}  % 表のキャプション
        \begin{tabular}{|c|c|c|c|}  % 列を定義
          \hline  % 横線
          クラス & 学習 & 検証 & テスト \\  % ヘッダー行
          \hline  % 横線
          normal & 9740 & 9650 & 9766 \\  % 1行目
          \hline  % 横線
          dos & 39127 & 39238 & 39106 \\  % 2行目
          \hline  % 横線
          probe & 412 & 385 & 410 \\  % 3行目
          \hline  % 横線
          r2l & 120 & 125 & 115 \\  % 4行目
          \hline  % 横線
          u2r & 3 & 4 & 5 \\  % 5行目
          \hline  % 横線
          合計 & 49402 & 49402 & 49402 \\  % 合計行
          \hline  % 横線
        \end{tabular}
        \label{3kdd99}  % 表のラベル 
    \end{minipage}
  \end{table}
本研究では,ABCで得られたSVMモデルが未知のデータに対して有効であるかを評価するために
3つのデータセットを用意した\cite{origin}.3つのデータセットは,
SVMの学習に使用する学習セット,SVMの評価に使用する検証セット,
最適解を評価するためのテストセットである.
ここでテストセットはABCで得られた最適解を評価するために一度だけ使用される.
3つのデータセットを内訳を表~\ref{3kdd99}に示す.
これらのデータセットは10\%KDD'99データセットからランダムに10\%抽出している.
\subsection{実験設定}
\subsubsection{実験環境}
SVMの実装には,CPU:Intel Core i7-12700,メモリ32GBの計算機上で
Python~3.11.0とscikit-learn~1.5.0ライブラリのSVCクラスを使用した.
SVCクラスにおける本研究で使用するハイパーパラメータのデフォルト設定を表~\ref{default}に示す.
ここでn\_featureはデータセットの特徴数を表す.
\begin{table}[tb]
    \centering
    \caption{SVCクラスのデフォルト設定}
    \begin{tabular}{|c|c|}  % 2列を定義
      \hline  % 横線
      パラメータ & デフォルト値 \\  % ヘッダー行    
      \hline  % 横線
      kernel & rbf\\  % 2行目
      \hline  % 横線
      $C$ & 1.0 \\  % 1行目
      \hline  % 横線     
      $\gamma$ & 1/n\_feature\\  % 3行目
      \hline  % 横線
      coef0 & 0\\  % 4行目
      \hline  % 横線
      degree($d$) & 3\\  % 4行目
      \hline  % 横線
  \end{tabular}
     % 表のキャプション
    \label{default}  % 表のラベル 
  \end{table}
\subsubsection{パラメータ設定}
表~\ref{abc:param},\ref{svm:param}に本研究のパラメータ設定を示す
\begin{table}[tb]
    \centering
    \begin{minipage}{0.45\textwidth}  % 左側の表の幅
      \centering
      \caption{ABCの実験パラメータ}  % 表のキャプション
      \begin{tabular}{|c|c|}  % 2列を定義
        \hline  % 横線
        コロニーサイズ & 20 \\  % ヘッダー行
        \hline  % 横線
        LIMIT & 100 \\  % 1行目
        \hline  % 横線
        サイクル数 & 500 \\  % 2行目
        \hline  % 横線
      \end{tabular}
      \label{abc:param}  % 表のラベル 
    \end{minipage}
    \begin{minipage}{0.45\textwidth}  % 右側の表の幅
      \centering
      \caption{SVMの実験パラメータ}  % 表のキャプション
      \begin{tabular}{|c|c|}  % 2列を定義
        \hline  % 横線
        kernel & [linear,rbf,sigmoid,poly] \\  % 2行目
        \hline  % 横線
        $C$ & [$10^{-6}$,35000] \\  % 1行目
        \hline  % 横線     
        $\gamma$ & [$10^{-6}$,32] \\  % 3行目
        \hline  % 横線
        coef0 & [0,10] \\  % 4行目
        \hline  % 横線
        degree($d$) & [1,3] \\  % 4行目
        \hline  % 横線
      \end{tabular}
      \label{svm:param}  % 表のラベル 
    \end{minipage}
  \end{table}
\subsection{実験結果}
実験結果を表~\ref{result}に示す.
\begin{table}[h]
    \centering
    \caption{実験結果}  % 表のキャプション
    \begin{tabular}{|c|c|c|c|c|c|c|}  % 3列を定義(c: 中央揃え、|: 縦線)
        \hline  % 横線
        ~ & 線形 &rbf &シグモイド&多項式&既存手法 & 提案手法\\  % ヘッダー行
        \hline  % 横線
        分類精度[\%]& 99.68&99.78&96.12&99.76&99.84& 99.90\\  % 1行目
        \hline  % 横線
        実行時間[h] & - & -&-&-&11.7& 7.0\\  % 2行目
        \hline  % 横線
    \end{tabular}
   
    \label{result}  % 表のラベル
  \end{table}