\section{実験}
侵入検知問題である10\%KDD'99データセットを使用して,
scikit-learnのデフォルトパラメータ,先行研究,提案手法の間で比較実験を行った.
先行研究はカーネル関数をRBFカーネルに固定し,
ABCアルゴリズムを使用してSVMの$C$と$\gamma$の最適化と特徴選択を行っている\cite{origin}.
\subsection{データセット}
10\%KDD'99データセットは,オリジナルのKDD'99データセットのうちサンプル数の多いnormal,
dos,probeクラスを10\%に減らしたデータセットである.また,KDD'99データセットの特徴数は41個である.
KDD'99データセットの内訳を表~\ref{kdd99}に示す.
\begin{table}[tb]
    \centering
    \begin{minipage}{0.45\textwidth}  % 表の幅を指定
        \centering
        \caption{KDD'99データセットの内訳}  % 表のキャプション
        \begin{tabular}{|l|r|r|}  % 左揃えと右揃えに変更
          \hline  % 横線
          クラス & KDD'99 & 10\%KDD'99 \\  % ヘッダー行
          \hline  % 横線
          normal & 972780 & 97279 \\  % 1行目
          \hline  % 横線
          dos & 3883370 & 391458 \\  % 2行目
          \hline  % 横線
          probe & 41102 & 4107 \\  % 3行目
          \hline  % 横線
          r2l & 1126 & 1126 \\  % 4行目
          \hline  % 横線
          u2r & 52 & 52 \\  % 5行目
          \hline  % 横線
          合計 & 4898430 & 494021 \\  % 合計行
          \hline  % 横線
        \end{tabular}
        \label{kdd99}  % 表のラベル 
    \end{minipage}
    \begin{minipage}{0.45\textwidth}  % 表の幅を指定
        \centering
        \caption{実験データセットの内訳}  % 表のキャプション
        \begin{tabular}{|c|c|c|c|}  % 列を定義
          \hline  % 横線
          クラス & 学習 & 検証 & テスト \\  % ヘッダー行
          \hline  % 横線
          normal & 9740 & 9650 & 9766 \\  % 1行目
          \hline  % 横線
          dos & 39127 & 39238 & 39106 \\  % 2行目
          \hline  % 横線
          probe & 412 & 385 & 410 \\  % 3行目
          \hline  % 横線
          r2l & 120 & 125 & 115 \\  % 4行目
          \hline  % 横線
          u2r & 3 & 4 & 5 \\  % 5行目
          \hline  % 横線
          合計 & 49402 & 49402 & 49402 \\  % 合計行
          \hline  % 横線
        \end{tabular}
        \label{3kdd99}  % 表のラベル 
    \end{minipage}
  \end{table}
本研究では,ABCで得られたSVMモデルが未知のデータに対して有効であるかを評価するために
3つのデータセットを用意した\cite{origin}.3つのデータセットは,
SVMの学習に使用する学習セット,SVMの評価に使用する検証セット,
最適解を評価するためのテストセットである.
ここでテストセットはABCで得られた最適解を評価するために一度だけ使用される.
3つのデータセットを内訳を表~\ref{3kdd99}に示す.
これらのデータセットは10\%KDD'99データセットからランダムに10\%抽出している.
\subsection{実験設定}
\subsubsection{実験環境}
SVMの実装には,CPU:Intel Core i7-12700,メモリ32GBの計算機上で
Python~3.11.0とscikit-learn~1.5.0ライブラリのSVCクラスを使用した.
SVCクラスにおける本研究で使用するハイパーパラメータのデフォルト設定を表~\ref{default}に示す.
ここでn\_featureはデータセットの特徴数を表す.本研究では$\text{n\_feature}=38$である.
\begin{table}[tb]
    \centering
    \caption{SVCクラスのデフォルト設定}
    \begin{tabular}{|c|c|}  % 2列を定義
      \hline  % 横線
      パラメータ & デフォルト値 \\  % ヘッダー行    
      \hline  % 横線
      kernel & RBF\\  % 2行目
      \hline  % 横線
      $C$ & 1.0 \\  % 1行目
      \hline  % 横線     
      $\gamma$ & 1/n\_feature\\  % 3行目
      \hline  % 横線
      coef0 & 0\\  % 4行目
      \hline  % 横線
      $d$ & 3\\  % 4行目
      \hline  % 横線
  \end{tabular}
     % 表のキャプション
    \label{default}  % 表のラベル 
  \end{table}
\subsubsection{パラメータ設定}
表~\ref{abc:param},\ref{svm:param}に本研究のパラメータ設定を示す.
表~\ref{abc:param}のABCにおけるパラメータは先行研究を参照し,
両手法ともに同じパラメータ設定である\cite{origin}.
\ref{svm:param}のSVMのパラメータは先行研究では$C$,$\gamma$のみを扱う.
提案手法で新たに扱う4つのkernelは,SVCクラスにあるすべてのカーネル関数である.
coef0,$d$の探索範囲は試行を重ねた上で調節した.
\begin{table}[tb]
    \centering
    \begin{minipage}{0.4\textwidth}  % 左側の表の幅
      \centering
      \caption{ABCの実験パラメータ}  % 表のキャプション
      \begin{tabular}{|c|c|}  % 2列を定義
        \hline  % 横線
        コロニーサイズ & 20 \\  % ヘッダー行
        \hline  % 横線
        LIMIT & 100 \\  % 1行目
        \hline  % 横線
        サイクル数 & 500 \\  % 2行目
        \hline  % 横線
      \end{tabular}
      \label{abc:param}  % 表のラベル 
    \end{minipage}
    \begin{minipage}{0.5\textwidth}  % 右側の表の幅
      \centering
      \caption{SVMの実験パラメータ}  % 表のキャプション
      \begin{tabular}{|c|c|}  % 2列を定義
        \hline  % 横線
        kernel & [linear,RBF,sigmoid,poly] \\  % 2行目
        \hline  % 横線
        $C$ & [$10^{-6}$,35000] \\  % 1行目
        \hline  % 横線     
        $\gamma$ & [$10^{-6}$,32] \\  % 3行目
        \hline  % 横線
        coef0 & [0,10] \\  % 4行目
        \hline  % 横線
        $d$ & [1,3] \\  % 4行目
        \hline  % 横線
      \end{tabular}
      \label{svm:param}  % 表のラベル 
    \end{minipage}
  \end{table}
\subsection{評価指標}
本研究では,得られたSVMモデルの性能評価のため5つのクラスに対する分類精度に加えて,
検知率,誤警報率,適合率,F値の4つの評価指標を算出する.
これら4つの指標を得るために必要な値を以下に示す.
ここで,通常状態のクラスはnormal,攻撃状態のクラスはdos,probe,r2l,u2rである.
\begin{itemize}
    \item 真陽性(TP):攻撃状態として分類され,実際に攻撃状態だったデータ数
    \item 真陰性(TN):通常状態として分類され,実際に通常状態だったデータ数
    \item 偽陽性(FP):攻撃状態として分類され,実際には通常状態だったデータ数
    \item 偽陰性(FN):通常状態として分類され,実際には攻撃状態だったデータ数
\end{itemize}
TP,TN,FP,FNによって検知率,誤警報率,適合率,F値
はそれぞれ(\ref{detect})〜(\ref{Fvalue})式で算出される.

\setlength{\jot}{12pt}
\begin{align}
\text{検知率} &= \dfrac{TP}{TP + FN}\label{detect}\\
\text{誤警報率} &= \dfrac{FP}{TN + FP}\\
\text{適合率} &= \dfrac{TP}{TP + FP}\\
\text{F値} &= \dfrac{2\text{*検知率*適合率}}{\text{検知率}+\text{適合率}}\label{Fvalue}
\end{align}

検知率は,攻撃状態を検知できた割合.
誤警報率は,通常状態のデータを攻撃状態と分類した割合.
適合率は,攻撃状態として分類され実際に攻撃状態だった割合.
F値は検知率と適合率の調和平均である.
\subsection{実験結果}
以下に示す先行研究と提案手法に関する結果はすべて,それぞれ10回ずつ実行した平均値である.
各カーネル関数におけるSVCクラスのデフォルトパラメータ,先行研究,提案手法で比較実験を行った結果を
表~\ref{result1}に示す.ただし,分類精度は小数点第3位を,実行時間は小数点第2位を四捨五入している.
また,線形,RBF,シグモイド,多項式はカーネル関数を指定し,SVCクラスのデフォルト
パラメータを使用したものである.
加えて,先行研究と提案手法のTP,TN,FP,FNの値を表~\ref{result2}に,
表~\ref{result2}から算出した検知率,誤警報率,適合率,F値を表~\ref{result3}に示す.
ただし,表~\ref{result2}の値は小数点第2位を,
表~\ref{result3}の値は小数点第3位を四捨五入している.
また,表~\ref{featuregood}〜\ref{myparam1}に,
先行研究と提案手法で求まった最良解と最悪解を示す.
\begin{table}[b]
    \centering
    \caption{実験結果}  % 表のキャプション
    \begin{tabular}{|c|c|c|c|c|c|c|}  % 3列を定義(c: 中央揃え,|: 縦線)
        \hline  % 横線
        ~ & 線形 &RBF &シグモイド&多項式&先行研究 & 提案手法\\  % ヘッダー行
        \hline  % 横線
        分類精度[\%]& 99.68&99.78&96.12&99.76&99.88& 99.91\\  % 1行目
        \hline  % 横線
        実行時間[h] & - & -&-&-&11.8& 15.5\\  % 2行目
        \hline  % 横線
    \end{tabular}
    \label{result1}  % 表のラベル
  \end{table}
  \begin{table}[h]
    \centering
    \begin{minipage}{0.45\textwidth}  % 幅を指定
        \centering
        \caption{混同行列の値}  % 表のキャプション
        \begin{tabular}{|c|c|c|}  % 3列を定義(c: 中央揃え,|: 縦線)
            \hline  % 横線
            ~ &先行研究 & 提案手法\\  % ヘッダー行
            \hline  % 横線
            TP & 39602.4 & 39609.7\\  % 1行目
            \hline  % 横線
            TN & 9743.8 & 9748.9\\  % 2行目
            \hline  % 横線
            FP & 22.2 & 17.1\\  % 3行目
            \hline  % 横線
            FN & 33.6 & 26.3\\  % 4行目
            \hline  % 横線
        \end{tabular}
        \label{result2}  % 表のラベル
    \end{minipage} \hspace{1cm}  % 両表の間隔
    \begin{minipage}{0.45\textwidth}  % 幅を指定
        \centering
        \caption{性能評価指標の結果}  % 表のキャプション
        \begin{tabular}{|c|c|c|}  % 3列を定義(c: 中央揃え,|: 縦線)
            \hline  % 横線
            ~ &先行研究 & 提案手法\\  % ヘッダー行
            \hline  % 横線
            検知率[\%] & 99.91 & 99.93\\  % 1行目
            \hline  % 横線
            誤警報率[\%] & 0.23 & 0.17\\  % 2行目
            \hline  % 横線
            適合率[\%] & 99.94 & 99.96\\  % 3行目
            \hline  % 横線
            F値[\%] & 99.93 & 99.95\\  % 4行目
            \hline  % 横線
        \end{tabular}
        \label{result3}  % 表のラベル
    \end{minipage}
  \end{table}
  \begin{table}[htbp]
  \setlength{\tabcolsep}{2pt} % 列間の間隔を調整
  \renewcommand{\arraystretch}{0.8} % 行間を狭くする
  \centering
  \caption{特徴選択の結果(最良解:特徴数23)}
  \begin{tabular}{|l|c|l|c|}
    \hline
    特徴名 & 選択の有無 & 特徴名 & 選択の有無   \\
    \hline
    duration & \(\circ\) & count & \(\circ\)  \\
    \hline
    src\_bytes &  \(\times\) & srv\_count & \(\circ\)   \\
    \hline
    dst\_bytes & \(\circ\) & serror\_rate & \(\circ\)   \\
    \cline{1-4}
    land & \(\times\) & srv\_serror\_rate & \(\circ\)   \\
    \cline{1-4}
    wrong\_fragment & \(\circ\) & rerror\_rate & \(\times\)   \\
    \cline{1-4}
    urgent & \(\circ\) & srv\_rerror\_rate & \(\circ\)   \\
    \cline{1-4}
    hot & \(\circ\) & same\_srv\_rate & \(\times\)   \\
    \cline{1-4}
    num\_failed\_logins & \(\circ\) & diff\_srv\_rate &  \(\circ\)  \\
    \cline{1-4}
    logged\_in & \(\circ\) & srv\_diff\_host\_rate &  \(\circ\)   \\
    \cline{1-4}
    num\_compromised & \(\circ\) & dst\_host\_count &\(\circ\)   \\
    \cline{1-4}
    root\_shell &\(\times\) & dst\_host\_srv\_count &\(\circ\)   \\
    \cline{1-4}
    su\_attempted &\(\times\) & dst\_host\_same\_srv\_rate & \(\circ\)   \\
    \cline{1-4}
    num\_root & \(\times\) & dst\_host\_diff\_srv\_rate &  \(\circ\)  \\
    \cline{1-4}
    num\_file\_creations &  \(\times\) & dst\_host\_same\_src\_port\_rate &   \(\times\)   \\
    \cline{1-4}
    num\_shells &  \(\times\) & dst\_host\_srv\_diff\_host\_rate & \(\circ\)  \\
    \cline{1-4}
    num\_access\_files &  \(\times\) & dst\_host\_serror\_rate & \(\times\)    \\
    \cline{1-4}
    num\_outbound\_cmds & \(\times\) & dst\_host\_srv\_serror\_rate & \(\circ\)   \\
    \cline{1-4}
    is\_host\_login & \(\circ\) &  dst\_host\_rerror\_rate & \(\times\)    \\
    \cline{1-4}
    is\_guest\_login & \(\times\) & dst\_host\_srv\_rerror\_rate & \(\circ\)  \\
    \cline{1-4}
\end{tabular}
\label{featuregood} 
\end{table}
\begin{table}[htbp]
  \setlength{\tabcolsep}{2pt} % 列間の間隔を調整
  \renewcommand{\arraystretch}{0.8} % 行間を狭くする
  \centering
  \caption{特徴選択の結果(最悪解:特徴数21)}
  \begin{tabular}{|l|c|l|c|}
      \hline
      特徴名 & 選択の有無 & 特徴名 & 選択の有無   \\
      \hline
      duration & \(\circ\) & count & \(\circ\)  \\
      \hline
      src\_bytes & \(\circ\) & srv\_count & \(\times\)   \\
      \hline
      dst\_bytes & \(\times\) & serror\_rate & \(\circ\)   \\
      \cline{1-4}
      land & \(\times\) & srv\_serror\_rate & \(\times\)   \\
      \cline{1-4}
      wrong\_fragment & \(\circ\) & rerror\_rate & \(\times\)   \\
      \cline{1-4}
      urgent & \(\circ\) & srv\_rerror\_rate & \(\times\)   \\
      \cline{1-4}
      hot & \(\circ\) & same\_srv\_rate & \(\times\)   \\
      \cline{1-4}
      num\_failed\_logins & \(\circ\) & diff\_srv\_rate &  \(\times\)  \\
      \cline{1-4}
      logged\_in & \(\circ\) & srv\_diff\_host\_rate &  \(\times\)   \\
      \cline{1-4}
      num\_compromised & \(\times\) & dst\_host\_count & \(\times\)   \\
      \cline{1-4}
      root\_shell & \(\circ\) & dst\_host\_srv\_count &\(\circ\)   \\
      \cline{1-4}
      su\_attempted & \(\circ\) & dst\_host\_same\_srv\_rate & \(\circ\)   \\
      \cline{1-4}
      num\_root & \(\times\) & dst\_host\_diff\_srv\_rate &  \(\circ\)  \\
      \cline{1-4}
      num\_file\_creations &  \(\times\) & dst\_host\_same\_src\_port\_rate &  \(\circ\)   \\
      \cline{1-4}
      num\_shells &  \(\times\) & dst\_host\_srv\_diff\_host\_rate & \(\times\)   \\
      \cline{1-4}
      num\_access\_files &  \(\times\) & dst\_host\_serror\_rate & \(\circ\)   \\
      \cline{1-4}
      num\_outbound\_cmds & \(\times\) & dst\_host\_srv\_serror\_rate & \(\circ\)   \\
      \cline{1-4}
      is\_host\_login & \(\circ\) &  dst\_host\_rerror\_rate & \(\circ\)   \\
      \cline{1-4}
      is\_guest\_login & \(\circ\) & dst\_host\_srv\_rerror\_rate & \(\circ\)  \\
      \cline{1-4}
  \end{tabular}
  \label{featurebad} 
\end{table}
\clearpage
\begin{table}[tb]
  \centering
  \caption{最良解と最悪解(先行研究)}  % 表のキャプション
  \begin{tabular}{|c|c|c|c|}  % 3列を定義(c: 中央揃え,|: 縦線)
      \hline  % 横線
      &$C$ &$\gamma$ &分類精度[\%] \\  % ヘッダー行
      \hline  % 横線
    最良解 &7550.20& 0.255&99.897 \\  % 1行目
      \hline  % 横線
    最悪解  &35000.00&15.81 &99.848\\  % 2行目
      \hline  % 横線
  \end{tabular}
  \label{myparam2}  % 表のラベル
\end{table}
\begin{table}[tb]
  \centering
  \caption{最良解と最悪解(提案手法)}  % 表のキャプション
  \begin{tabular}{|c|c|c|c|}  % 3列を定義(c: 中央揃え,|: 縦線)
      \hline  % 横線
      &$C$ &$\gamma$ &分類精度[\%] \\  % ヘッダー行
      \hline  % 横線
    最良解&469.44&0.485&99.915\\  % 1行目
      \hline  % 横線
    最悪解 &95.63 &0.731&99.907\\  % 2行目
      \hline  % 横線
  \end{tabular}
  \label{myparam1}  % 表のラベル
\end{table}