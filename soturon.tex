\documentclass[12pt, dvipdfmx]{jarticle}
\usepackage{ilsfonts}    % 卒論スタイルで使うフォントの定義
\usepackage{ilssoturon}  % 卒論スタイル宣言
\usepackage[dvips]{graphicx} % グラフィクスの利用宣言
\usepackage{amsmath,amssymb} % 数学記号などの利用宣言
\usepackage{bm}  % 太字数式文字の利用に関する宣言
%\usepackage{slashbox} % 表に斜線を入れる(不要かも)
%\usepackage[dvips]{graphicx,psfrag} % 図に数式を使うときに用いる(別の方法がおすすめ)
%\usepackage[dvipdfmx]{graphicx} %図
% ハイパーリンクを作るためのパッケージだが,pdf viewerによってはハイパーリンクの枠が描画
% されたまま出力されるので,基本的には使用しない
%\usepackage[dvipdfmx]{hyperref}
%\usepackage{pxjahyper} % 日本語しおりを作るためのパッケージ
%% 以下は論文表紙に出力される内容
\年度{2024年度}
\提出日{2024年1月xx日}
\研究者{2131007 &安達 拓真}
\論文題目{Artificial Bee Colonyアルゴリズムによる\\サポートベクトルマシン
の\\ハイパーパラメータ最適化}

\begin{document} % ここから論文本体
\卒論表紙 % これで数式が出る

% 以下の4行は目次を定義している
% 正しく目次を出力するためには tex コンパイルを2回連続でかけること
\pagenumbering{roman} % 目次のページ番号はローマ数字
\tableofcontents % 目次を出す命令
\clearpage % 改ページ
\pagenumbering{arabic} % ページ番号をアラビア数字になおす

% ここからが論文本体。ただし、訂正がしやすいように節ごとにファイルを
% 分割している。ここに論文本体を入れてもコンパイルは可能だか、
% けっしておすすめできない。
\section{はじめに}
機械学習モデルには,あらかじめ決めておかなければいけない値であるハイパーパラメータが
存在する.
これらのハイパーパラメータはモデルの性能に大きな影響を与えるため,
適切なハイパーパラメータの選択が必要不可欠である\cite{essential}.
ハイパーパラメータの例として,分類や回帰に用いられるサポートベクトルマシン(SVM)では,
ペナルティパラメータ$C$やカーネル関数のパラメータが挙げられる.
これらのハイパーパラメータは離散、連続、カテゴリなど様々である.
そのため,ハイパーパラメータ最適化(Hyper Parameter Optimization, HPO)
では,高次元かつ複雑な探索空間の探索が必要である.
さらに,目的関数を評価するためにモデルの学
習が必要となるため,
多くの場合目的関数の評価が実行時間におけるボト
ルネックとなる.
そのためHPOでは評価回数と実行時間がトレードオフの関係にある\cite{trade}.

従来のハイパーパラメータ調整は手動調整やグリッドサーチ,ランダムサーチで行われてきた.
手動でハイパーパラメータを調節することは直感と経験に頼る作業になり,
グリッドサーチ,ランダムサーチでは自動化されたものの,
探索効率が悪く,高次元の探索空間では計算コストが大きな課題となる.

これらの課題を解決するため,より効率的な探索手法として,群知能が注目されている.
群知能とは,自然界の生物の群れが高度な振る舞いをすることをコンピュータに適用したアルゴリズムである\cite{population}.
群知能の代表的な手法には,人工蜂コロニー(ABC),粒子群最適化(PSO),蟻コロニー最適化(ACO)
などがある.
特にABCは,設定パラメータが少なく比較的シンプルなアルゴリズムであるため,
HPOの最適化アルゴリズムとして適している.

先行研究では,ABCを用いてSVMのハイパーパラメータ最適化と特徴選択を行い,
SVMのカーネル関数を固定して実験を行っていた\cite{origin}.
しかしSVMには様々なカーネル関数が適用でき,それぞれハイパーパラメータが異なる.
そのため,カーネル関数の選択をハイパーパラメータとして扱うことは,
より良いSVMモデルの探索を可能にする可能性がある.

そこで本研究では,
先行研究で固定されていたカーネル関数を含むハイパーパラメータ全体を探索対象とし,
ABCアルゴリズムを用いてSVMのハイパーパラメータ最適化を行う.
具体的には,$C$,4種類のカーネル関数(線形,RBF,シグモイド,多項式),
およびそのカーネル関数に対応するパラメータを最適化対象とする.
これにより,SVMの分類性能をさらに向上させることを目的とする.
 本研究では,10\%KDD99データセットを使用して提案手法と先行研究との比較実験を行った.
その結果提案手法で得られたパラメータセットは,先行研究で得られたものよりも分類精度が高いという結果となった.
混同行列の評価も書く.
以降の構成は,2章と3章にてSVMとABCの理論を示し,4章で提案手法の説明を行なう.
5章で実験,6章で考察.
 % intro.tex を読み込む。
\clearpage % 改ページ(節が終るごとに改ページしてください)

\謝辞 % ここから謝辞
山口先生有り難う! % 謝辞のないよう
%\clearpage

%\begin{thebibliography}{99} % 参考文献をここに書く(TeXの参考書を参照)
%\addcontentsline{toc}{section}{参考文献}
%\bibitem{yam}山口 智, \ 
%"Differential Evolutionにおける制御変数の自動調整`` \  電気学会論文誌 \ 電子・情報・システム部門誌, Vol.128 NO.11 , pp.1696-1703 \ (2008)
%\end{thebibliography} % 参考文献環境の終了
\end{document} % 本文の終了
