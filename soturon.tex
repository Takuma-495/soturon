\documentclass[12pt, dvipdfmx]{jarticle}
\usepackage{ilsfonts}    % 卒論スタイルで使うフォントの定義
\usepackage{ilssoturon}  % 卒論スタイル宣言
\usepackage[dvips]{graphicx} % グラフィクスの利用宣言
\usepackage{amsmath,amssymb} % 数学記号などの利用宣言
\usepackage{bm}  % 太字数式文字の利用に関する宣言
\numberwithin{equation}{section}
%\usepackage{slashbox} % 表に斜線を入れる(不要かも)
%\usepackage[dvips]{graphicx,psfrag} % 図に数式を使うときに用いる(別の方法がおすすめ)
%\usepackage[dvipdfmx]{graphicx} %図
% ハイパーリンクを作るためのパッケージだが,pdf viewerによってはハイパーリンクの枠が描画
% されたまま出力されるので,基本的には使用しない
%\usepackage[dvipdfmx]{hyperref}
%\usepackage{pxjahyper} % 日本語しおりを作るためのパッケージ
%% 以下は論文表紙に出力される内容
\年度{2024年度}
\提出日{2024年12月25日}
\研究者{2131007 &安達 拓真}
\論文題目{Artificial Bee Colonyアルゴリズムによる\\サポートベクターマシン
の\\ハイパーパラメータ最適化}

\begin{document} % ここから論文本体
\卒論表紙 % これで数式が出る

% 以下の4行は目次を定義している
% 正しく目次を出力するためには tex コンパイルを2回連続でかけること
\pagenumbering{roman} % 目次のページ番号はローマ数字
\tableofcontents % 目次を出す命令
\clearpage % 改ページ
\pagenumbering{arabic} % ページ番号をアラビア数字になおす

% ここからが論文本体。ただし,訂正がしやすいように節ごとにファイルを
% 分割している。ここに論文本体を入れてもコンパイルは可能だか,
% けっしておすすめできない。
\section{はじめに}
機械学習モデルには,あらかじめ決めておかなければいけない値であるハイパーパラメータが
存在する.
これらのハイパーパラメータはモデルの性能に大きな影響を与えるため,
適切なハイパーパラメータの選択が必要不可欠である\cite{essential}.
ハイパーパラメータの例として,分類や回帰に用いられるサポートベクトルマシン(SVM)では,
ペナルティパラメータ$C$やカーネル関数のパラメータが挙げられる.
これらのハイパーパラメータは離散、連続、カテゴリなど様々である.
そのため,ハイパーパラメータ最適化(Hyper Parameter Optimization, HPO)
では,高次元かつ複雑な探索空間の探索が必要である.
さらに,目的関数を評価するためにモデルの学
習が必要となるため,
多くの場合目的関数の評価が実行時間におけるボト
ルネックとなる.
そのためHPOでは評価回数と実行時間がトレードオフの関係にある\cite{trade}.

従来のハイパーパラメータ調整は手動調整やグリッドサーチ,ランダムサーチで行われてきた.
手動でハイパーパラメータを調節することは直感と経験に頼る作業になり,
グリッドサーチ,ランダムサーチでは自動化されたものの,
探索効率が悪く,高次元の探索空間では計算コストが大きな課題となる.

これらの課題を解決するため,より効率的な探索手法として,群知能が注目されている.
群知能とは,自然界の生物の群れが高度な振る舞いをすることをコンピュータに適用したアルゴリズムである\cite{population}.
群知能の代表的な手法には,人工蜂コロニー(ABC),粒子群最適化(PSO),蟻コロニー最適化(ACO)
などがある.
特にABCは,設定パラメータが少なく比較的シンプルなアルゴリズムであるため,
HPOの最適化アルゴリズムとして適している.

先行研究では,ABCを用いてSVMのハイパーパラメータ最適化と特徴選択を行い,
SVMのカーネル関数を固定して実験を行っていた\cite{origin}.
しかしSVMには様々なカーネル関数が適用でき,それぞれハイパーパラメータが異なる.
そのため,カーネル関数の選択をハイパーパラメータとして扱うことは,
より良いSVMモデルの探索を可能にする可能性がある.

そこで本研究では,
先行研究で固定されていたカーネル関数を含むハイパーパラメータ全体を探索対象とし,
ABCアルゴリズムを用いてSVMのハイパーパラメータ最適化を行う.
具体的には,$C$,4種類のカーネル関数(線形,RBF,シグモイド,多項式),
およびそのカーネル関数に対応するパラメータを最適化対象とする.
これにより,SVMの分類性能をさらに向上させることを目的とする.
 本研究では,10\%KDD99データセットを使用して提案手法と先行研究との比較実験を行った.
その結果提案手法で得られたパラメータセットは,先行研究で得られたものよりも分類精度が高いという結果となった.
混同行列の評価も書く.
以降の構成は,第2章でSVMの説明,第3章でABCの説明,第4章で既存手法の説明,第5章で提案手法の説明を行う.
第6章では本研究で行った実験の説明と結果を示し,第7章では実験結果からの考察を行う.
最後に,第8章では本研究の総括と今後の展望について検討する.
 % intro.tex を読み込む。
\clearpage % 改ページ(節が終るごとに改ページしてください)
\section{サポートベクトルマシン(SVM)}
内容1
 % intro.tex を読み込む。
\clearpage
\section{Artificial Bee Colonyアルゴリズム(ABC)}
\subsection{概要}
郡知能とは個々での単純な行動が,集団としては高度な振る舞いをすることを模した最適化アルゴリズムである.
そしてABCは2005年にkaraboraによって提案された郡知能の一つである\cite{abc}.
ABCは蜜蜂の採餌行動から着想を得ており,収穫蜂,追従蜂,偵察蜂の3種類の蜂によって探索を行う.
収穫蜂はすべての食物源に対して探索を行い,追従蜂は評価値の高い食物源を優先して探索を行なう.
追従蜂の探索によって評価値の高い食物源の近傍を局所的に探索することができる.
収穫蜂と追従蜂による探索で改善されなかった食物源は偵察蜂によってランダムな新しい食物源に置き換えられる.
偵察蜂によって局所最適解からの脱出が可能になり,探索範囲を広げることができる.
これら3種類の蜂による探索を繰り返すことで,より良い解を求めていく.
\subsection{探索手順}
ABCによって,(\ref{problem})式のような目的関数を$f(\boldsymbol{x})$を最小とするD次元の$\boldsymbol{x}$を求める問題を解くことを考える.
\begin{align}
    \label{problem}
\text{min}.~f(\boldsymbol{x}), \text{~~subject to } \boldsymbol{x} \in \mathbb{R} ^D
\end{align}
ABCでは食物源の数$n$と探索上限回数limitの2つのパラメータをあらかじめ設定する必要がある.
食物源の数を$n$とすると,収穫蜂,追従蜂の数はそれぞれ$n$ずつとなる.
limitは偵察蜂フェーズで使用する食物源の探索回数の上限である.
この値を小さくするとより広範囲の探索に,大きくするとより局所的な探索を行う.
各食物源に番号を振り分け,$i$番目の食物源を$\boldsymbol{x_{i}}(i = 1,...,n)$とすると
$\boldsymbol{x_{i}}$の評価値$\mathrm{fit}(\boldsymbol{x_{i}})$は
(\ref{eva})式の評価関数により求められる.
\begin{align}
    \label{eva} 
    \mathrm{fit}(\boldsymbol{x_{i}}) =
    \begin{cases}
    \dfrac{1}{1+f(\boldsymbol{x_{i}})} & \text{if } f(\boldsymbol{x_{i}}) \geq 0   \\
    \left\lvert1+f(\boldsymbol{x_{i}})\right\rvert & otherwise
    \end{cases}
\end{align}
\subsubsection*{step 0 準備}
ABCにおけるパラメータである食物源の数$n$と探索上限回数limitの設定を行う.次に終了条件の設定を行なう.
終了条件としては,アルゴリズムの反復回数が最大反復回数を超えたとき,
最良の食物源の評価値もしくは目的関数値が一定値を超えたときなどがある.
\subsubsection*{step 1 初期化}
初期化では,(\ref{init})式によって各食物源を各成分の範囲内でランダムに生成する.
さらに各食物源の探索回数$trial_i$を0に初期化する.
$x_{ij}$は$\boldsymbol{x_{i}}$の$j(j=1,...,D)$番目の成分, 
$x_{\min_j}$,$x_{\max_j}$はそれぞれ食物源の各成分の最小値,最大値である.また,$\mathrm{rand}$は一様乱数を表す. 
\begin{align}
    x_{ij} = x_{\min_j} + \mathrm{rand}[0,1](x_{\max_j}-x_{\min_j})\label{init}
\end{align}
初期化された食物源の中で最も評価値が高いものを$\boldsymbol{x}_{best}$とし,
その食物源のインデックスを$i_{best}$とする
\subsubsection*{step 2 収穫蜂フェーズ}
収穫蜂フェーズでは収穫蜂が食物源の近傍を探索する.
このときの更新式は(\ref{harvest})式のようになる.
ここで$j,k(k\neq i)$はランダムに選択される.
\begin{align}
v_{ij} = x_{ij} + \mathrm{rand}[-1,1](x_{ij}-x_{kj})\label{harvest}
\end{align}
ここで$v_{ij}$と$x_{ij}$の評価値を比較し,$v_{ij}$が$x_{ij}$よりも優れていたら$x_{ij}$を$v_{ij}$に置き換え,
$trial_i$を0にリセットする.
そうでなければ,$trial_i$を1増やす.
\subsubsection*{step 3 追従蜂フェーズ}
追従蜂フェーズでは収穫蜂フェーズによって更新された食物源の評価値に基づいて,
ルーレット選択を行い更新個体を選択する.
食物源$x_i$の選択確率$p_i$は(\ref{roulette})式のようになる.
そのため,評価値が高い食物源ほど選択確率が高く,評価値が低い食物源は選択確率が低くなる.
食物源の更新は収穫蜂フェーズと同様に行われる.
この操作を$n$回行う.
\begin{align}
    p_i = \dfrac{\mathrm{fit}(\boldsymbol{x_{i}})}{\sum_{n}\mathrm{fit}(\boldsymbol{x_{j}})}\label{roulette}
\end{align}
\subsubsection*{step 4 偵察蜂フェーズ}
偵察蜂フェーズでは,trialの値が探索打ち切り回数limitに達した
食物源を(\ref{init})式によりランダムに生成した新たな解に置換する. 
置換した食物源のtrialを0にリセットする.
trialの値がlimitに達した
食物源がなければ何も行わない.
\subsubsection*{step 5 終了判定}
各食物源の評価値$\mathrm{fit}(\boldsymbol{x_{i}})$と,$\mathrm{fit}(\boldsymbol{x}_{best})$を比較し,
$\mathrm{fit}(\boldsymbol{x_{i}}) > \mathrm{fit}(\boldsymbol{x}_{best})$となる$\boldsymbol{x_{i}}$が存在する場合は
$\boldsymbol{x}_{best}$を更新する.
その後,終了条件を満たしている場合は最適解をその時の$\boldsymbol{x}_{best}$として探索を終了し,
そうでない場合は\textbf{step 2}に戻り,探索を続ける.
 % intro.tex を読み込む。
\clearpage
\section{既存手法}
先行研究として,ABCアルゴリズムを用いて
SVMのハイパーパラメータ最適化と特徴選択を行った研究がある\cite{origin}.
先行研究では,カーネル関数をrbfに固定してSVMのパラメータ$C$と,
rbfのパラメータ$\gamma$の2つのパラメータを最適化対象としている.
さらに特徴選択を行うため,データセットの特徴数を$N$とすると個体表現は(\ref{ori_expression})の$N+2$次元
である.
\begin{align}
  \text{個体表現:} \quad  (\text{feature1,feature2,...,feature}N,~C,~\gamma)\label{ori_expression}
\end{align}
そして,各成分の値の範囲はすべて[0,1]であり,
feature1,feature2,...,feature$N$に関しては0.5以上でその特徴を使用し0.5未満では使用しない.
$C$と$\gamma$に関しては,成分の値を$a$とすると
(\ref{map})式によってSVMに適用される値に変換される.
\begin{align}
  \label{map}
  A =a(a_{max} -a_{min}) + a_{min}
\end{align}
ここで$a_{max}$,$a_{min}$はパラメータの上限値,下限値を表している.
この方法により各成分の値の範囲をすべて[0,1]で表すことができるため実装が簡素になる
ABCにおける食物源の評価値$\mathrm{fit}(\boldsymbol{x_{i}})$は(\ref{fitness})式によって計算される.
ここで$\mathrm{miss}(\boldsymbol{x_{i}})$はSVMが分類を行い,誤分類した割合である.
\begin{align}
    \label{fitness}
    \mathrm{fit}(\boldsymbol{x_{i}}) = \dfrac{1}{1+ \mathrm{miss}(\boldsymbol{x_{i}})}
\end{align} % intro.tex を読み込む。
\clearpage
\section{提案手法}
SVMのハイパーパラメータ最適化では,カーネル関数の選択も考慮する余地があるが,カーネル関数を固定して行われた研究が多い.
本研究では,SVMのハイパーパラメータ$C$に加え,
(\ref{k1})〜(\ref{k4})式の4つのカーネル関数とそのハイパーパラメータ($\gamma$,coef0,$d$)
を最適化対象とする手法を提案する.
最適化アルゴリズムは設定パラメータの少ないABCを使用する.
これにより,ハイパーパラメータの探索範囲を広げ,より良いSVMモデルの探索が可能になる.
\subsection{ABCアルゴリズムの適用}
(\ref{k1})〜(\ref{k4})式の4つのカーネル関数は,ハイパーパラメータの数や性質が異なっている.
そのためカーネル関数が異なる個体では個体表現が変わり,個体の次元数が変化する.
しかし,(\ref{k2})〜(\ref{k4})式の$\gamma$や,
(\ref{k3}),(\ref{k4})式のcoef0はそれぞれスケール項,定数項と共通の性質を持っている.
そこで,本研究ではABCの個体表現を(\ref{expression})の5次元とし,
個体表現の変化をその個体の各次元の活性,非活性により表現することとした.
ここで,カーネル関数,$C$は常に使用されるハイパーパラメータで
$\gamma$,coef0,$d$はその個体のカーネル関数によって活性,非活性になる.
\begin{align}
    \text{個体表現:} \quad  (\text{カーネル関数},~C,~\gamma,~\text{coef0},~d)\label{expression}
\end{align}
カーネル関数の値は4つのカーネル関数を整数値にエンコードしたもので,
$C,~\gamma,~\text{coef0},~d$は[0.1]の範囲の実数とした.
得られた実数を$a$とするとそれぞれのパラメータは(\ref{map})式によって
指定した範囲内でSVMに適用する値$A$となる\cite{origin}.
ここで$a_{max}$,$a_{min}$はパラメータの上限値,下限値を表している.
\begin{align}
    \label{map}
    A =a(a_{max} -a_{min}) + a_{min}
\end{align}
また,$d$は離散値であるため,$A$を四捨五入したものをSVMに適用し,非活性の次元は探索の際に選択されない.
表~\ref{tab:param}にカーネル関数による活性,非活性の状態の組み合わせを示す.ここで1が活性,0が非活性を表す.
\begin{table}[t]
    \centering
    \begin{tabular}{|c|c|c|c|}  % 3列を定義(c: 中央揃え、|: 縦線)
        \hline  % 横線
        カーネル関数 & $\gamma$ & coef0 & $d$\\  % ヘッダー行
        \hline  % 横線
        線形カーネル& 0& 0& 0\\  % 1行目
        \hline  % 横線
        RBFカーネル & 1 & 0& 0\\  % 2行目
        \hline  % 横線
        シグモイドカーネル & 1 & 1& 0\\  % 2行目
        \hline  % 横線
        多項式カーネル & 1 & 1& 1\\  % 2行目
        \hline  % 横線
    \end{tabular}
    \caption{カーネル関数によるパラメータの活性,非活性}  % 表のキャプション
    \label{tab:param}  % 表のラベル
  \end{table} % 表のラベル
  
また,ABCによる最適化は連続値を前提としているため,
カテゴリ変数であるカーネル関数にABCの更新式は適用できない.
そこで本研究では更新次元にカーネル関数が選択された際の更新は
(\ref{karnel_update})式の確率$P$で更新されるものとする.
ここで$\boldsymbol{x_i}$は更新個体,
$\boldsymbol{x_j}$は$\boldsymbol{x_i}$と異なるカーネル関数を持つ個体の中からランダムに選ばれた個体である.
\begin{align}
    \label{karnel_update}
   P = \dfrac{f(\boldsymbol{x_j})}{f(\boldsymbol{x_i})+f(\boldsymbol{x_j})}
\end{align}

ABCにおける食物源の評価値$fit_i$は(\ref{fitness})式によって計算される.
ここで$miss_i$はSVMが検証セットの分類を行い,誤分類した割合である.
\begin{align}
    \label{fitness}
    fit_i = \dfrac{1}{1+miss_i}
\end{align}
\subsection{提案手法のアルゴリズム}
q % intro.tex を読み込む。
\clearpage
\section{実験}
侵入検知問題である10\%KDD'99データセットを使用して,
scikit-learnのデフォルトパラメータ,既存手法,提案手法の間で比較実験を行った.
既存手法はカーネル関数をrbfに固定し,
ABCアルゴリズムを使用してSVMの$C$と$\gamma$の最適化と特徴選択を行っている\cite{origin}.
\subsection{データセット}
10\%KDD'99データセットは,オリジナルのKDD'99データセットのうちサンプル数の多いnormal,
dos,probeクラスを10\%に減らしたデータセットである.また,KDD'99データセットの特徴数は41個である.
KDD'99データセットの内訳を表~\ref{kdd99}に示す.
\begin{table}[tb]
    \centering
    \begin{minipage}{0.45\textwidth}  % 表の幅を指定
        \centering
        \caption{KDD'99データセットの内訳}  % 表のキャプション
        \begin{tabular}{|l|r|r|}  % 左揃えと右揃えに変更
          \hline  % 横線
          クラス & KDD'99 & 10\%KDD'99 \\  % ヘッダー行
          \hline  % 横線
          normal & 972780 & 97279 \\  % 1行目
          \hline  % 横線
          dos & 3883370 & 391458 \\  % 2行目
          \hline  % 横線
          probe & 41102 & 4107 \\  % 3行目
          \hline  % 横線
          r2l & 1126 & 1126 \\  % 4行目
          \hline  % 横線
          u2r & 52 & 52 \\  % 5行目
          \hline  % 横線
          合計 & 4898430 & 494021 \\  % 合計行
          \hline  % 横線
        \end{tabular}
        \label{kdd99}  % 表のラベル 
    \end{minipage}
    \begin{minipage}{0.45\textwidth}  % 表の幅を指定
        \centering
        \caption{実験データセットの内訳}  % 表のキャプション
        \begin{tabular}{|c|c|c|c|}  % 列を定義
          \hline  % 横線
          クラス & 学習 & 検証 & テスト \\  % ヘッダー行
          \hline  % 横線
          normal & 9740 & 9650 & 9766 \\  % 1行目
          \hline  % 横線
          dos & 39127 & 39238 & 39106 \\  % 2行目
          \hline  % 横線
          probe & 412 & 385 & 410 \\  % 3行目
          \hline  % 横線
          r2l & 120 & 125 & 115 \\  % 4行目
          \hline  % 横線
          u2r & 3 & 4 & 5 \\  % 5行目
          \hline  % 横線
          合計 & 49402 & 49402 & 49402 \\  % 合計行
          \hline  % 横線
        \end{tabular}
        \label{3kdd99}  % 表のラベル 
    \end{minipage}
  \end{table}
本研究では,ABCで得られたSVMモデルが未知のデータに対して有効であるかを評価するために
3つのデータセットを用意した\cite{origin}.3つのデータセットは,
SVMの学習に使用する学習セット,SVMの評価に使用する検証セット,
最適解を評価するためのテストセットである.
ここでテストセットはABCで得られた最適解を評価するために一度だけ使用される.
3つのデータセットを内訳を表~\ref{3kdd99}に示す.
これらのデータセットは10\%KDD'99データセットからランダムに10\%抽出している.
\subsection{実験設定}
\subsubsection{実験環境}
SVMの実装には,CPU:Intel Core i7-12700,メモリ32GBの計算機上で
Python~3.11.0とscikit-learn~1.5.0ライブラリのSVCクラスを使用した.
SVCクラスにおける本研究で使用するハイパーパラメータのデフォルト設定を表~\ref{default}に示す.
ここでn\_featureはデータセットの特徴数を表す.本研究では$\text{n\_feature}=38$である.
\begin{table}[tb]
    \centering
    \caption{SVCクラスのデフォルト設定}
    \begin{tabular}{|c|c|}  % 2列を定義
      \hline  % 横線
      パラメータ & デフォルト値 \\  % ヘッダー行    
      \hline  % 横線
      kernel & rbf\\  % 2行目
      \hline  % 横線
      $C$ & 1.0 \\  % 1行目
      \hline  % 横線     
      $\gamma$ & 1/n\_feature\\  % 3行目
      \hline  % 横線
      coef0 & 0\\  % 4行目
      \hline  % 横線
      $d$ & 3\\  % 4行目
      \hline  % 横線
  \end{tabular}
     % 表のキャプション
    \label{default}  % 表のラベル 
  \end{table}
\subsubsection{パラメータ設定}
表~\ref{abc:param},\ref{svm:param}に本研究のパラメータ設定を示す.
表~\ref{abc:param}のABCにおけるパラメータは既存手法を参照し,
両手法同じパラメータ設定である\cite{origin}.
\ref{svm:param}のSVMのパラメータは既存手法では$C$,$\gamma$のみを扱う.
提案手法で新たに扱う4つのkernelは,SVCクラスにあるすべてのカーネル関数である.
coef0,$d$の探索範囲は試行を重ねた上で調節した.


\begin{table}[tb]
    \centering
    \begin{minipage}{0.45\textwidth}  % 左側の表の幅
      \centering
      \caption{ABCの実験パラメータ}  % 表のキャプション
      \begin{tabular}{|c|c|}  % 2列を定義
        \hline  % 横線
        コロニーサイズ & 20 \\  % ヘッダー行
        \hline  % 横線
        LIMIT & 100 \\  % 1行目
        \hline  % 横線
        サイクル数 & 500 \\  % 2行目
        \hline  % 横線
      \end{tabular}
      \label{abc:param}  % 表のラベル 
    \end{minipage}
    \begin{minipage}{0.45\textwidth}  % 右側の表の幅
      \centering
      \caption{SVMの実験パラメータ}  % 表のキャプション
      \begin{tabular}{|c|c|}  % 2列を定義
        \hline  % 横線
        kernel & [linear,rbf,sigmoid,poly] \\  % 2行目
        \hline  % 横線
        $C$ & [$10^{-6}$,35000] \\  % 1行目
        \hline  % 横線     
        $\gamma$ & [$10^{-6}$,32] \\  % 3行目
        \hline  % 横線
        coef0 & [0,10] \\  % 4行目
        \hline  % 横線
        $d$ & [1,3] \\  % 4行目
        \hline  % 横線
      \end{tabular}
      \label{svm:param}  % 表のラベル 
    \end{minipage}
  \end{table}
\subsection{評価指標}
\subsection{実験結果}
実験結果を表~\ref{result}に示す.
\begin{table}[h]
    \centering
    \caption{実験結果}  % 表のキャプション
    \begin{tabular}{|c|c|c|c|c|c|c|}  % 3列を定義(c: 中央揃え、|: 縦線)
        \hline  % 横線
        ~ & 線形 &rbf &シグモイド&多項式&既存手法 & 提案手法\\  % ヘッダー行
        \hline  % 横線
        分類精度[\%]& 99.68&99.78&96.12&99.76&99.84& 99.89\\  % 1行目
        \hline  % 横線
        実行時間[h] & - & -&-&-&11.7& 13.6\\  % 2行目
        \hline  % 横線
    \end{tabular}
   
    \label{result}  % 表のラベル
  \end{table} % intro.tex を読み込む。
\clearpage
\section{考察}
内容1

\clearpage
\section{おわりに}
内容1
 % intro.tex を読み込む。
\clearpage

\謝辞 % ここから謝辞
卒業研究および卒業論文の執筆を行うにあたり,
終始ご指導を賜りました山口智教授に心より感謝申し上げます.
そして,研究活動を行うにあたり有益な助言をくださった院生の先輩方
,山口研究室の方々に心より御礼申し上げます% 謝辞のないよう
\clearpage

\begin{thebibliography}{99} % 参考文献をここに書く(TeXの参考書を参照)
\addcontentsline{toc}{section}{参考文献}
\bibitem{essential}M. Feurer and F. Hutter, “Hyperparameter optimization,” Automated Machine Learning, pp.3–33, Springer, 2019.
\bibitem{trade}尾崎 嘉彦,野村 将寛,大西 正輝,“ 機械学習におけるハイ
パパラメータ最適化手法:概要と特徴”,電子情報通信学
会論文誌,vol.J103-D, No.9, pp.615-631, Sep. 2020.
\bibitem{population}井上 一哉,鈴木 麻里子,“群知能によるパラメータ最適化”,
土木学会論文集 A2(応用力学), Vol. 74, No. 2 (応用力学論文集 Vol. 21), I\_33-I\_44, 2018.
\bibitem{origin}近藤 久,浅沼 由馬“人工蜂コロニーアルゴリズムによるランダムフォレストとサポートベクトルマシンのハイパーパラメータ最適化と特徴選択”,
人工知能学会論文誌, vol34-2, pp.1-11, 2019.
\bibitem{svm}Cortes, C. and Vapnik, V. Support-vector networks, Ma-chine Learning, Vol.20, No.3, pp.273-297, 1995.
\bibitem{abc}Karaboga, Dervis. An idea based on honey bee swarm for numerical optimization. Vol. 200. 
Technical report-tr06, Erciyes university, engineering faculty, computer engineering department, 2005.
\end{thebibliography} % 参考文献環境の終了
\end{document} % 本文の終了
