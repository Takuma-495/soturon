\section{サポートベクトルマシン(SVM)}
SVMは1995年にCortesらによって提案された機械学習アルゴリズムである[\cite{svm}].
SVMは教師あり学習を用いる2値分類器であるが、多クラスへの拡張も可能である.
SVMは,訓練サンプル集合からデータを分類する識別関数を学習するアルゴリズムである.
この学習過程において,SVMは訓練データの中で識別関数に近いデータであるサポートベクトルを得る.
そして,サポートベクトルと識別関数との距離,すなわちマージンを最大化するように識別関数を構築する.
これにより,SVMは新しいデータを分類する際,
最大限のマージンを持ってデータを分類できるようになり,
未知のデータに対しても誤分類のリスクを最小限に抑えることができる.
\subsection{ハードマージンSVM}
$n$個の$N$次元教師データ$\boldsymbol{x}_i(i=1,...,n)$,
がクラス$K_1$,$K_2$の2値分類データであるとして,
対応するクラスラベルをクラス$K_1$の時に$y_i= 1$,
クラス$K_2$の時に$y_i= -1$,とするとデータが線形分離可能であれば,
識別関数は(\ref{decision})式のように定義される.
 $\boldsymbol{w}$は$m$次元ベクトル, $b$は識別関数のパラメータである.
\begin{align}
    \label{decision}
f(\boldsymbol{x}) = \boldsymbol{w}^T \boldsymbol{x} + b
\end{align}
すべての教師データ$\boldsymbol{x}_i(i=1,...,n)$について(\ref{learn})式の条件を満たすように
$\boldsymbol{w}$, $b$を調節する事がSVMの学習フェーズとなる.
\begin{align}
    \label{learn}
    f(\boldsymbol{x}) = \boldsymbol{w}^T \boldsymbol{x}_i + b =
    \begin{cases}
        \geq 1&  \boldsymbol{x}_i \in K_1 \\
        \leq  -1& \boldsymbol{x}_i \in K_2
    \end{cases}
\end{align}
このとき,(\ref{learn})式は(\ref{equal})式と等価になる.
\begin{align}
    \label{equal}
    y_i(\boldsymbol{w}^T \boldsymbol{x}_i + b) \geq 1
\end{align}
ここにサポートベクトルと超平面(線)の図\\
ここで図??にデータを分離する超平面(2次元では直線)を示す.
分離超平面とそれに最も近いデータ(サポートベクトル)との距離をマージンと呼び,
SVMはマージンを最大化するように分離超平面を求める.
マージンを最大化することで,
サポートベクトルから少しずれたデータが存在しても分類可能になり,
モデルの汎化性能が向上する.
次に,マージンを最大化する超平面を求める方法を導く.
$\boldsymbol{x}_i$から超平面への距離を$d$とすると,
$d$は(\ref{distance})式のようになる.
\begin{align}
    \label{distance}
d = \dfrac{|f(\boldsymbol{x_i})|}{\|\boldsymbol{w}\| }
\end{align}
よって,マージンを$M$とすると,
マージン最大化は(\ref{margin})式の最適化問題によって表現できる
\begin{align}
    \label{margin}
    \underset{{\boldsymbol{w},b}}{\text{max}}~M,~~ \text{subject to } \dfrac{y_i(\boldsymbol{w}^T \boldsymbol{x}_i + b)}{\|\boldsymbol{w}\|}
    \geq M ~~  (i=1,...,n)
\end{align}
さらに,(\ref{margin})式の制約条件部分を$M$で割り,
$\frac{\boldsymbol{w}}{M\|\boldsymbol{w}\|}$と
$\frac{b}{M\|\boldsymbol{w}\|}$
をそれぞれ$\tilde{w}$,$\tilde{b}$
とすると制約条件は(\ref{chmargin})式に書き換えられる.
\begin{align}
    \label{chmargin}
    y_i(\boldsymbol{\tilde{w}}^T \boldsymbol{x}_i + \tilde{b}) \geq 1
\end{align}
(\ref{chmargin})式の等号が成立するデータ
$\boldsymbol{x_i}$は分離超平面と距離が最も近いデータ,
すなわちサポートベクトルである.
したがってマージン$\tilde{M}$は(\ref{newmargin})式のようになる
\begin{align}
    \label{newmargin}
   \tilde{M} = \dfrac{1}{\|\boldsymbol{\tilde{w}}\|}
\end{align}
よって(\ref{margin})式は(\ref{finalmargin})式で表すことができる.
\begin{align}
    \label{finalmargin}
    \underset{\boldsymbol{\tilde{w}},\tilde{b}}{\text{max}}~\dfrac{1}{\|\boldsymbol{\tilde{w}}\|},
  ~~ \text{subject to } y_i(\boldsymbol{\tilde{w}}^T \boldsymbol{x}_i + \tilde{b}) \geq 1~~  (i=1,...,n)
\end{align}
$\|\boldsymbol{\tilde{w}}\|$は正であるため,
$\frac{1}{\|\boldsymbol{\tilde{w}}\|}$を最大化する問題は,
${\|\boldsymbol{\tilde{w}}\|}^2$を最小化する問題に等しい.
$\boldsymbol{\tilde{w}}$,$\tilde{b}$を改めて$\boldsymbol{w}$,$b$と表記すると
(\ref{finalmargin})式は(\ref{minmargin})式に書き換えられ,
マージン最大化は(\ref{minmargin})式を解く問題に帰着できる.
\begin{align}
    \label{minmargin}
    \underset{\boldsymbol{w},b}{\text{min}}~\dfrac{1}{2}{\|\boldsymbol{{w}}\|}^2,
  ~~ \text{subject to } y_i(\boldsymbol{w}^T \boldsymbol{x}_i + b) \geq 1~~  (i=1,...,n)
\end{align}
さらに,制約付き最適化問題である(\ref{minmargin})式を
ラグランジュの未定乗数法により,双対問題に置き換える.
ラグランジュ乗数$\boldsymbol{\alpha} = (\alpha_1,...,\alpha_n)^T$を用いて
ラグランジュ関数$L$を(\ref{lag})式と定義する.
\begin{align}
    \label{lag}
    L(\boldsymbol{w},b,\boldsymbol{\alpha}) 
    = \dfrac{1}{2}{\|\boldsymbol{{w}}\|}^2
    - \sum_{n = 1}^{n} \alpha_i \{y_i(\boldsymbol{w}^T \boldsymbol{x}_i + b)-1\}
\end{align}
制約条件が不等式であるため,$\boldsymbol{w}$の最適解のおいて以下のKKT条件が適用できる.
\begin{subequations}
\begin{align}
   \frac{\partial L(\boldsymbol{w},b,\boldsymbol{\alpha})}{\partial \boldsymbol{w}} = 0\label{w}\\
    \frac{\partial L(\boldsymbol{w},b,\boldsymbol{\alpha})}{\partial \boldsymbol{b}} = 0\label{b}\\
    y_i(\boldsymbol{w}^T \boldsymbol{x}_i + b)-1 = 0\label{Support}\\
    \alpha_i \geq 0
\end{align}
\end{subequations}
(\ref{Support})式より,$\alpha_i = 0$または$\alpha_i > 0$で$y_i(\boldsymbol{w}^T \boldsymbol{x}_i + b)=1$
が満たされなければならない.ここで,$\alpha_i > 0$となる$\boldsymbol{x_i}$をサポートベクトルと呼ぶ.
SVMは,教師データ中から,サポートベクトルのみを用いて識別関数を構成する.
 (\ref{w})式, (\ref{b})式を計算すると,それぞれ(\ref{w2})式, (\ref{b2})式になる.
 \begin{subequations}
 \begin{align}
   \boldsymbol{w} = \sum_{i=1}^{n}\alpha_i y_i \boldsymbol{w_i} \label{w2}\\
   \sum_{i=1}^{n}\alpha_i y_i = 0 \label{b2}
 \end{align}
\end{subequations}
(\ref{w2})式, (\ref{b2})式を(\ref{lag})式に代入すると,
$\boldsymbol{\alpha}$のみで表されるハードマージンの双対問題が得られる.
これは線形分離可能な場合のみに適用でき,誤分類を許容しない.
\begin{align}
    \underset{\boldsymbol{a}}{\text{max}} \left\{\tilde{L}(\boldsymbol{\alpha}) 
    = \sum_{i=1}^{n}\alpha_i - \dfrac{1}{2}\sum_{i,j=1}^{n}
    \alpha_i\alpha_j y_i y_j \boldsymbol{x_i}^T \boldsymbol{x_j}\right\} \nonumber \\
    \text{subject to} \sum_{i=1}^{n}\alpha_i y_i = 0, \alpha_i \geq 0,\ (i=1,...,n)
\end{align}
ここで,サポートベクトルに対応する添字の集合をSとすると,(\ref{w2})式より,
識別関数は(\ref{s.decision})式になる.
\begin{align}
    \label{s.decision}
    D(\boldsymbol{x}) = \sum_{i\in S}a_iy_i\boldsymbol{x_i}^T\boldsymbol{x} + b
\end{align}
また,(\ref{Support})式より,$b$は(\ref{finalb})式になる.
\begin{align}
  \label{finalb}
  b =y_i - \boldsymbol{x_i}^T\boldsymbol{x}(i \in S)
\end{align}
(\ref{s.decision}),(\ref{finalb})より得られた識別関数により,
$\boldsymbol{x}$は$D(\boldsymbol{x}) > 0$ならクラス$K_1$に,
$D(\boldsymbol{x}) < 0$ならクラス$K_2$に分類する.

\subsection{ソフトマージンSVM}
2.1節で述べたハードマージンSVMは線形分離可能なことを前提としているが,現実の問題は非線形な問題が多い.
そこで非線形な問題に対しても適応できるように,(\ref{equal})式に非負の変数$\xi_i(\geq 0)$を導入する.
その式を(\ref{soft})式に示す.
\begin{align}
    \label{soft}
    y_i(\boldsymbol{w}^T \boldsymbol{x}_i + b) \geq 1 - \xi_i
\end{align}
これにより,マージンの内側にデータが存在することを許容する.
このときソフトマージンの最適化問題は,(\ref{soft1})式,(\ref{soft2})式になる.
ここで$C$は誤分類をどれだけ許容するかを決めるハイパーパラメータであり,
小さいほど誤分類を許容し,大きいほど誤分類を許容しない.よって
この問題は,(\ref{soft1})式の第1項のマージン最大化と
第2項の誤分類の許容数のバランスを図る問題である.
\begin{align}
    \underset{\boldsymbol{w},b,\boldsymbol{\xi}}{\text{min}}~\dfrac{1}{2}{\|\boldsymbol{{w}}\|}^2
    &+C\sum_{i=1}^{n}\xi_i \label{soft1}\\
   \text{subject to } y_i(\boldsymbol{w}^T \boldsymbol{x}_i + b) \geq &1 - \xi_i,\xi_i \geq 0 ~(i=1,...,n)\label{soft2}
\end{align}
2.1節と同様に,(\ref{soft1})式,(\ref{soft2})式を主問題として,ラグランジュの未定乗数法により双対問題を導出する.
ラグランジュ乗数$\boldsymbol{\alpha} = (\alpha_1,...,\alpha_n)^T$,
$\boldsymbol{\beta} = (\beta_1,...,\beta_n)^T$を用いて
ラグランジュ関数$L$を(\ref{slag})式と定義する.
\begin{align}
    \label{slag}
    L(\boldsymbol{w},b,\boldsymbol{\xi},\boldsymbol{\alpha},\boldsymbol{\beta}) 
    = \dfrac{1}{2}{\|\boldsymbol{{w}}\|}^2+C\sum_{i=1}^{n}\xi_i
    - \sum_{n = 1}^{n} \alpha_i \{y_i(\boldsymbol{w}^T \boldsymbol{x}_i + b)-1+\xi_i\}
    -\sum_{i=1}^{n}\beta_i\xi_i
\end{align}
制約条件が不等式であるため,$\boldsymbol{w}$の最適解のおいて以下のKKT条件が適用できる.
\begin{subequations}
\begin{align}
   \frac{\partial L(\boldsymbol{w},b,\boldsymbol{\xi},\boldsymbol{\alpha},\boldsymbol{\beta})}{\partial \boldsymbol{w}} = 0\label{sw}\\
    \frac{\partial L(\boldsymbol{w},b,\boldsymbol{\xi},\boldsymbol{\alpha},\boldsymbol{\beta})}{\partial \boldsymbol{b}} = 0\label{sb}\\
    \frac{\partial L(\boldsymbol{w},b,\boldsymbol{\xi},\boldsymbol{\alpha},\boldsymbol{\beta})}{\partial \boldsymbol{\xi}} = 0\label{sxi}\\
    \alpha_i \{y_i(\boldsymbol{w}^T \boldsymbol{x}_i + b)-1+\xi_i\}=0\label{sSupport}\\
    \beta_i\xi_i = 0\label{bxi}\\
    \alpha_i \geq 0 ,\beta_i \geq 0,\xi_i \geq 0 
\end{align}
\end{subequations}
 (\ref{sw})〜(\ref{sxi})式を計算すると,それぞれ(\ref{sw2})〜(\ref{sb2})式になる.
 \begin{subequations}
 \begin{align}
   \boldsymbol{w} = \sum_{i=1}^{n}\alpha_i y_i \boldsymbol{w_i} \label{sw2}\\
   \sum_{i=1}^{n}\alpha_i y_i = 0 \\
   \alpha_i + \beta_i = C\label{sb2}
 \end{align}
\end{subequations}
(\ref{sw2})〜(\ref{sb2})式を(\ref{slag})式に代入すると,
$\boldsymbol{\alpha}$のみで表されるソフトマージンの双対問題が得られる.
\begin{align}
    \underset{\boldsymbol{a}}{\text{max}} \left\{\tilde{L}(\boldsymbol{\alpha}) 
    = \sum_{i=1}^{n}\alpha_i - \dfrac{1}{2}\sum_{i,j=1}^{n}
    \alpha_i\alpha_j y_i y_j \boldsymbol{x_i}^T \boldsymbol{x_j}\right\} \nonumber \\
    \text{subject to} \sum_{i=1}^{n}\alpha_i y_i = 0,0 \leq \alpha_i \leq C,\ (i=1,...,n)
\end{align}
識別関数はハードマージンSVMと同じく(\ref{s.decision}),(\ref{finalb})式になる.
ただし,(\ref{sSupport}),(\ref{bxi}),(\ref{sb2})式により,$\alpha_i$に関して次の3つの場合がある.


\subsection{カーネルトリック}
カーネルトリック
d次元特徴ベクトルをd`次元特徴空間に写像する

→このときの関数
本論文で使うカーネル関数を列挙

データの正規化

一対一法と一対他法について

→一対他法を使う
\subsection{多クラス分類への拡張}


