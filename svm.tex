\section{サポートベクトルマシン(SVM)}
SVMは1995年にCortesらによって提案された機械学習アルゴリズムである[\cite{svm}].
SVMの概略図を図?に示す.
SVMは,訓練サンプル集合からデータを分類する識別関数を学習するアルゴリズムである.
この学習過程において,SVMは訓練データの中で識別関数に近いデータであるサポートベクトルを得る.
そして,サポートベクトルと識別関数との距離,すなわちマージンを最大化するように識別関数を構築する.
これにより,SVMは新しいデータを分類する際,
最大限のマージンを持ってデータを分類できるようになり,
未知のデータに対しても誤分類のリスクを最小限に抑えることができる.
\subsection{ハードマージンSVM}
2値分類の場合,線形な識別関数は(\ref{decision})式のように定義される.ここで$\boldsymbol{x}$は入力ベクトル,
 $\boldsymbol{w}$, $b$は識別関数のパラメータである.
\begin{align}
    \label{decision}
f(\boldsymbol{x}) = \boldsymbol{w}^T \boldsymbol{x} + b
\end{align}
クラス$K_1$,$K_2$を以下のデータの集合と定義し,それぞれ1, -1とラベル付けすると,
\begin{align*}
    K_1:\boldsymbol{w}^T \boldsymbol{x}_i + b > 0\\
    K_2:\boldsymbol{w}^T \boldsymbol{x}_i + b < 0
\end{align*}
すべての学習サンプル$\boldsymbol{x}_i(i=1,...,n)$について(\ref{learn})式の条件を満たすように
$\boldsymbol{w}$, $b$を調節する事がSVMの学習フェーズとなる.
\begin{align}
    \label{learn}
    f(\boldsymbol{x}) = \boldsymbol{w}^T \boldsymbol{x}_i + b =
    \begin{cases}
        \geq 1&  \boldsymbol{x}_i \in K_1 \\
        \leq  -1& \boldsymbol{x}_i \in K_2
    \end{cases}
\end{align}
誤分類を許容せずに決定境界を求める場合をハードマージン

許容する場合をソフトマージン

\subsection{ソフトマージンSVM}

d次元特徴ベクトルをd`次元特徴空間に写像する

→このときの関数
\subsecton{カーネルトリック}
カーネルトリック

本論文で使うカーネル関数を列挙

データの正規化

一対一法と一対他法について

→一対他法を使う



