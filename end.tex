\section{おわりに}
本研究では,カーネル関数を含むハイパーパラメータ全体を探索対象とし,
ABCアルゴリズムを用いてSVMのハイパーパラメータ最適化を行う手法を提案した.
ABCアルゴリズムでカーネル関数を扱うために新たな更新式と探索アルゴリズムを導入した.
実験はKDD'99データセットを学習セット,検証セット,テストセットの3つに分けて
使用し,SVCクラスのデフォルトパラメータ,先行研究,提案手法の間で比較実験を行った.
その結果,提案手法の実行時間は先行研究よりも長くなってしまったが,
分類精度の面で先行研究よりも良い結果を得ることができた.
また,検知率,誤警報率,適合率,F値の値も改善することができ,
侵入検知問題への性能も向上した.
このことから,提案手法は実行時間よりも
モデルの性能を重視する環境において有用であると言える.

今後の課題としては, KDD'99データセット以外のデータセットでも実験を行うことや
先行研究よりも長くなってしまった実行時間の短縮を図ることが挙げられる.
特にその他のデータセットでも実験を行うことは重要で,
提案手法はカーネル関数もハイパーパラメータとして扱ったため,
様々なデータセットにおいても良い性能を発揮することが期待できる.

本研究で扱ったKDD'99データセットは比較的大規模なデータセットであるため,
SVMの学習を打ち切ったり,多項式カーネルの次元数の探索範囲を低く設定したが,
小規模なデータセットにおいてはこれらの制約を設けることなくハイパーパラメータの最適化
ができると考えられる.





