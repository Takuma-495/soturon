\section{おわりに}
本研究では,カーネル関数を含むハイパーパラメータ全体を探索対象とし,
ABCを用いてSVMのハイパーパラメータ最適化を行う手法を提案した.
そこで,ABCでカーネル関数を扱うために新たな更新式とアルゴリズムを導入した.
実験はKDD'99データセットを学習セット,検証セット,テストセットの3つに分けて
使用し,SVCクラスのデフォルトパラメータ,先行研究,提案手法の間で比較実験を行った.
その結果,提案手法の実行時間は先行研究よりも長くなってしまったが,
分類精度の面で先行研究よりも良い結果を得ることができた.
また,検知率,誤警報率,適合率,F値の値も改善することができ,
侵入検知問題への性能も向上した.
このことから,提案手法は実行時間よりも
モデルの性能を重視する環境において有用であると言える.

今後の課題としては,KDD'99データセット以外のデータセットを用いた実験や,
先行研究より長くなった実行時間の短縮が挙げられる.
特に提案手法はカーネル関数をハイパーパラメータとして扱っており,
データセットに適したカーネル関数を選択できる.
このため,他のデータセットでも実験を行うことは,
提案手法がさまざまなデータセットで高い性能を発揮できるかを検証する上で重要である.

本研究で扱ったKDD'99データセットは比較的大規模なデータセットであるため,
SVMの学習を打ち切ったり,多項式カーネルの次元数の探索範囲を低く設定する必要があるなど,
リソースが限られていた.
しかし,小規模なデータセットにおいてはリソースの制約が軽減されるため,
より広い探索範囲で提案手法が適用できると考えられる.





