\section{提案手法}
SVMのハイパーパラメータ最適化では,カーネル関数の選択も考慮する余地があるが,カーネル関数を固定して行われた研究が多い.
本研究では,SVMのハイパーパラメータ$C$に加え,
(\ref{k1})〜(\ref{k4})式の4つのカーネル関数とそのハイパーパラメータ($\gamma$,coef0,$d$)
を最適化対象とする手法を提案する.
最適化アルゴリズムは設定パラメータの少ないABCを使用する.
これにより,ハイパーパラメータの探索範囲を広げ,より良いSVMモデルの探索が可能になる.
\subsection{ABCアルゴリズムの適用}
(\ref{k1})〜(\ref{k4})式の4つのカーネル関数は,ハイパーパラメータの数や性質が異なっている.
そのためカーネル関数が異なる個体では個体表現が変わり,個体の次元数が変化する.
しかし,(\ref{k2})〜(\ref{k4})式の$\gamma$や,
(\ref{k3}),(\ref{k4})式のcoef0はそれぞれスケール項,定数項と共通の性質を持っている.
そこで,本研究ではABCの個体表現を(\ref{expression})の5次元とし,
個体表現の変化をその個体の各次元の活性,非活性により表現することとした.
ここで,カーネル関数,$C$は常に使用されるハイパーパラメータで
$\gamma$,coef0,$d$はその個体のカーネル関数によって活性,非活性になる.
\begin{align}
    \text{個体表現:} \quad  (\text{カーネル関数},~C,~\gamma,~\text{coef0},~d)\label{expression}
\end{align}
カーネル関数の値は4つのカーネル関数を整数値にエンコードしたもので,
$C,~\gamma,~\text{coef0},~d$は[0.1]の範囲の実数とした.
得られた実数を$a$とするとそれぞれのパラメータは(\ref{map})式によって
指定した範囲内でSVMに適用する値$A$となる\cite{origin}.
ここで$a_{max}$,$a_{min}$はパラメータの上限値,下限値を表している.
\begin{align}
    \label{map}
    A =a(a_{max} -a_{min}) + a_{min}
\end{align}
また,$d$は離散値であるため,$A$を四捨五入したものをSVMに適用し,非活性の次元は探索の際に選択されない.
表~\ref{tab:param}にカーネル関数による活性,非活性の状態の組み合わせを示す.ここで1が活性,0が非活性を表す.
\begin{table}[t]
    \centering
    \begin{tabular}{|c|c|c|c|}  % 3列を定義(c: 中央揃え、|: 縦線)
        \hline  % 横線
        カーネル関数 & $\gamma$ & coef0 & $d$\\  % ヘッダー行
        \hline  % 横線
        線形カーネル& 0& 0& 0\\  % 1行目
        \hline  % 横線
        RBFカーネル & 1 & 0& 0\\  % 2行目
        \hline  % 横線
        シグモイドカーネル & 1 & 1& 0\\  % 2行目
        \hline  % 横線
        多項式カーネル & 1 & 1& 1\\  % 2行目
        \hline  % 横線
    \end{tabular}
    \caption{カーネル関数によるパラメータの活性,非活性}  % 表のキャプション
    \label{tab:param}  % 表のラベル
  \end{table} % 表のラベル
  
また,ABCによる最適化は連続値を前提としているため,
カテゴリ変数であるカーネル関数にABCの更新式は適用できない.
そこで本研究では更新次元にカーネル関数が選択された際の更新は
(\ref{karnel_update})式の確率$P$で更新されるものとする.
ここで$\boldsymbol{x_i}$は更新個体,
$\boldsymbol{x_j}$は$\boldsymbol{x_i}$と異なるカーネル関数を持つ個体の中からランダムに選ばれた個体である.
\begin{align}
    \label{karnel_update}
   P = \dfrac{f(\boldsymbol{x_j})}{f(\boldsymbol{x_i})+f(\boldsymbol{x_j})}
\end{align}

ABCにおける食物源の評価値$fit_i$は(\ref{fitness})式によって計算される.
ここで$miss_i$はSVMが検証セットの分類を行い,誤分類した割合である.
\begin{align}
    \label{fitness}
    fit_i = \dfrac{1}{1+miss_i}
\end{align}
\subsection{提案手法のアルゴリズム}
q