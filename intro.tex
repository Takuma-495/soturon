\section{はじめに}
機械学習モデルには,あらかじめ決めておかなければいけない値であるハイパーパラメータが
存在する.
これらのハイパーパラメータはモデルの性能に大きな影響を与えるため,
適切なハイパーパラメータの選択が必要不可欠である\cite{essential}.
ハイパーパラメータの例として,分類や回帰に用いられるサポートベクターマシン(SVM)では,
ペナルティパラメータ$C$やカーネル関数の選択やカーネル関数自身がもつパラメータが挙げられる.
これらのハイパーパラメータは離散,連続,カテゴリなど様々である.
そのため,ハイパーパラメータ最適化(Hyper Parameter Optimization, HPO)
では,高次元かつ複雑な探索空間の探索が必要である.
さらに,目的関数を評価するためにモデルの学
習が必要となるため,
多くの場合目的関数の評価が実行時間におけるボト
ルネックとなる.
そのためHPOでは評価回数と実行時間がトレードオフの関係にある\cite{trade}.

従来のハイパーパラメータ調整は手動調整やグリッドサーチ,ランダムサーチで行われてきた.
手動でハイパーパラメータを調節することは直感と経験に頼る作業になり,
グリッドサーチ,ランダムサーチでは自動化されたものの,
探索効率が悪く,高次元の探索空間では計算コストが大きな課題となる\cite{trade}.

これらの課題を解決するため,より効率的な探索手法として,群知能が注目されている.
群知能とは,自然界の生物の群れが高度な振る舞いをすることをコンピュータに適用したアルゴリズムである\cite{population}.
群知能の代表的な手法には,人工蜂コロニー(ABC),粒子群最適化(PSO),蟻コロニー最適化(ACO)
などがある.
特にABCは,設定パラメータが少なく比較的シンプルなアルゴリズムであるため,
HPOの最適化アルゴリズムとして適している.

近藤らの研究\cite{origin}では,カーネル関数をRBFカーネルに固定した上で,
ABCを用いてSVMのハイパーパラメータ最適化と特徴選択を行っていた.
しかしSVMには様々なカーネル関数が適用でき,それぞれハイパーパラメータが異なる.
そのため,カーネル関数の選択をハイパーパラメータとして扱うことは,
より良いSVMモデルの探索を可能にする可能性がある.

そこで本研究では,
先行研究で固定されていたカーネル関数を含むハイパーパラメータ全体を探索対象とし,
ABCアルゴリズムを用いてSVMのハイパーパラメータ最適化を行う.
具体的には,$C$,4種類のカーネル関数(線形,RBF,シグモイド,多項式),
およびそのカーネル関数に対応するパラメータを最適化対象とする.
これにより,SVMの分類性能をさらに向上させることを目的とする.

本研究では,KDD'99データセットを使用してSVCクラスのデフォルトパラメータ,
先行研究,提案手法との間で比較実験を行った.
その結果,先行研究,提案手法ともにデフォルトパラメータよりも良い分類精度が得られた.
そして,提案手法の実行時間は先行研究よりも長くなったが,
提案手法で得られたパラメータセットは,先行研究で得られたものよりも分類精度が高いという結果が得られた.
また,検知率,誤警報率,適合率,F値の値も改善することができ,
侵入検知問題に対する性能も向上した.

以降の構成は,第2章でSVMの説明,第3章でABCの説明,第4章で既存手法の説明,第5章で提案手法の説明を行う.
第6章では本研究で行った実験の説明と結果を示し,第7章では実験結果からの考察を行う.
最後に,第8章では本研究の総括と今後の展望について検討する.
