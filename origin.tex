\section{既存手法}
先行研究として,ABCアルゴリズムを用いて
SVMのハイパーパラメータ最適化と特徴選択を行った研究がある\cite{origin}.
先行研究では,カーネル関数をrbfに固定してSVMのパラメータ$C$と,
rbfのパラメータ$\gamma$の2つのパラメータを最適化対象としている.
さらに特徴選択を行うため,データセットの特徴数を$N$とすると個体表現は(\ref{ori_expression})の$N+2$次元
である.
\begin{align}
  \text{個体表現:} \quad  (\text{feature1,feature2,...,feature}N,~C,~\gamma)\label{ori_expression}
\end{align}
そして,各成分の値の範囲はすべて[0,1]であり,
feature1,feature2,...,feature$N$に関しては0.5以上でその特徴を使用し0.5未満では使用しない.
$C$と$\gamma$に関しては,成分の値を$a$とすると
(\ref{map})式によってSVMに適用される値に変換される.
\begin{align}
  \label{map}
  A =a(a_{max} -a_{min}) + a_{min}
\end{align}
ここで$a_{max}$,$a_{min}$はパラメータの上限値,下限値を表している.
この方法により各成分の値の範囲をすべて[0,1]で表すことができるため実装が簡素になる
ABCにおける食物源の評価値$\mathrm{fit}(\boldsymbol{x_{i}})$は(\ref{fitness})式によって計算される.
ここで$\mathrm{miss}(\boldsymbol{x_{i}})$はSVMが分類を行い,誤分類した割合である.
\begin{align}
    \label{fitness}
    \mathrm{fit}(\boldsymbol{x_{i}}) = \dfrac{1}{1+ \mathrm{miss}(\boldsymbol{x_{i}})}
\end{align}