\section{考察}
まず,表~\ref{result1}より,デフォルトパラメータで最も分類精度の高いカーネル関数はRBFの99.68\%であり,
先行研究と提案手法の分類精度は99.88\%,99.91\%と,両手法とも
デフォルトパラメータを上回る結果となった.
また,提案手法は,先行研究よりも0.03\%分類精度が向上している.
テストセットのデータ数は49402であるため,
0.03\%の分類精度の向上は新たに平均15個のデータを分類できるようになったことに相当する.
さらに,先行研究の分類精度が99.88\%であるため改善の余地は残り0.12\%である.
このうち0.03\%の精度向上はその25\%を占めている.
加えて,ネットワークの侵入検知においては一つの攻撃を見逃すことが大きな損失になりかねないため
0.03\%の精度向上は意味のあるものだと考える.
一方実行時間の面では提案手法は先行研究よりも約3.7時間劣る結果となった.
また,表~\ref{result2},より先行研究に比べて提案手法のTP,TNの値は向上し
FP,FNの値は減少した.
そのため,表~\ref{result3}では検知率,適合率,F値が向上し,誤検知率は減少した.
よって提案手法で得られたSVMモデルは,分類精度だけでなく
侵入検知問題に対する性能も向上した.

提案手法の実行時間が先行研究の実行時間よりも長くなってしまった原因について考察する.

第一に,特徴選択の有無が挙げられる.先行研究では特徴選択をしているが,提案手法を行っていない.
特徴選択によって,データセットの次元削減が行われるとカーネル関数の計算が低コストになり,
SVMの計算コストが低減する.
また,多くの特徴を削減した個体はモデルの精度が低下する可能性があるが,そのような個体は低次元の計算になるため,
計算時間は短くなる.ゆえに評価値の低い個体に対して多くの計算資源を使う可能性も低くなる.
そのため特徴選択を行ったほうが実行時間は短くなると考えられる.

第二に,提案手法ではRBFカーネル以外のカーネル関数も扱っている点である.
RBFカーネルはハイパーパラメータが1個であるが,シグモイドカーネルは2個,多項式カーネルは3個とRBFカーネルよりも多い,
そのためシグモイドカーネル,多項式カーネルはハイパーパラメータ空間が広く,計算コストの高い個体も生成されやすい.
また,線形カーネルは元の次元で内積計算を行うためRBFカーネルよりも計算コストは高い.
そのため,提案手法の最適化過程では,先行研究よりも評価コストの高い個体を多く評価する必要がある.
表~\ref{result1}のデフォルトパラメータの結果から,RBFカーネルと多項式カーネルの分類精度が高いことが分かるように.
提案手法の最適化過程ではRBFカーネルと多項式カーネルの個体がほとんどであったと考えられる.
そのため多項式カーネルの次元数の範囲をさらに拡張して実験を行えば,より実行時間が長くなると考えられる.

次に,提案手法で最終的に得られた最良のモデルがすべてRBFカーネルを使用していたことについて考察する.
本研究では先行研究がカーネル関数をRBFカーネルに固定していた点に着目し,
4つのカーネル関数をハイパーパラメータとして扱ったが,
最終的には先行研究で固定していたRBFカーネルの個体が最良個体として得られた.
そのため,探索範囲をRBFカーネルに固定している先行研究では,
提案手法よりも良い結果が得られる可能性がある.
しかし,結果として提案手法の方がより良いモデルを生成することに成功している.

この理由として,探索の多様性と局所解への陥りやすさの違いが考えられる.
先行研究では,RBFカーネルに固定しているため,すべての個体は平均して良い評価値を得る.
そして,局所解からの脱出方法は偵察蜂フェーズのみである.
これに対し提案手法では,カーネル関数が更新される際に,
他のカーネル関数で使っていたパラメータをそのまま使用する.
そのため,カーネル関数が更新される際には偵察蜂フェーズのようなランダム性が生じる.
これにより,探索の多様性と局所解からの脱出の面で先行研究よりも優れた手法となった
可能性が考えられる.

また,提案手法では個体表現はハイパーパラメータの組の5次元だが,
先行研究での個体表現はハイパーパラメータに加え,特徴量を含めた40次元である.
さらに,提案手法はカーネル関数が変わると探索空間が変化するため,より低次元の探索となる.

以上のことから両手法間で探索空間が大きく異なるため,先行研究の手法では
探索空間内を十分に探索できていなかった可能性も考えられる.