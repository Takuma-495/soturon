\section{考察}
まず,表~\ref{result1}より,デフォルトパラメータで最も分類精度の高いカーネル関数はRBFの99.68\%であり,
先行研究と提案手法の分類精度は99.88\%,99.91\%と両手法とも
デフォルトパラメータよりも良い分類精度を得ることができた.
そして,提案手法は分類精度の面では,先行研究よりも0.03\%良い結果が得られた.
一方実行時間の面では提案手法は先行研究よりも約3.7時間劣る結果となった.
また,表~\ref{result2},\ref{result3}より
検知率,誤警報率,適合率,F値のすべてで提案手法のほうが良いという結果になった.
そのため分類精度に加えて,
侵入検知問題に使用するモデルとしても提案手法のほうが良いパラメータを得ている.

提案手法の実行時間が先行研究の実行時間よりも長くなってしまった原因について考察する.
まず,提案手法と先行研究の大きな違いとして,特徴選択の有無が挙げられる.
特徴選択によって,データセットの次元削減が行われるとカーネル関数の計算が低コストになり,
SVMの計算コストが低減する.
また,多くの特徴を削減した個体はモデルの精度が低下する可能性があるが,そのような個体は低次元の計算になるため,
計算時間は短くなる.ゆえに評価値の低い個体に対して多くの計算資源を使う可能性も低くなる.
そのため特徴選択を行ったほうが実行時間は短くなると考えられる.
次に,提案手法ではRBFカーネル以外のカーネル関数も扱い,個体によってカーネル関数が異なる点である.
RBFカーネルはハイパーパラメータが1個であるが,シグモイドカーネルは2個,多項式カーネルは3個とRBFカーネルよりも多い,
そのためシグモイドカーネル,多項式カーネルはハイパーパラメータ空間が広く,計算コストの高い個体も生成されやすい.
また,線形カーネルは元の次元で内積計算を行うためRBFカーネルよりも計算コストは高い.
そのため,提案手法の最適化過程では,先行研究よりも評価コストの高い個体を多く評価する必要がある.
表~\ref{result1}のデフォルトパラメータの結果から,RBFカーネルと多項式カーネルの分類精度が高いことが分かるように.
提案手法の最適化過程ではRBFカーネルと多項式カーネルの個体がほとんどであった.
そのため多項式カーネルの次元数の範囲をさらに拡張して実験を行えば,より実行時間が長くなると考えられる.

次に,提案手法で最終的に得られた最良のモデルがすべてRBFカーネルを使用していたことについて考察する.
本研究では先行研究がカーネル関数をRBFカーネルに固定していた点に着目し,
4つのカーネル関数をハイパーパラメータとして扱ったが,
最終的には先行研究で固定していたRBFカーネルの個体が最良個体として得られた.
そのため,良い個体が生成されやすいRBFカーネルに固定している先行研究の方が
良い結果が得られるとも考えられる.
しかし,結果として提案手法の方がより良いモデルを生成することに成功している.
この理由として,探索の多様性と局所解への陥りやすさが考えられる.
先行研究では,RBFカーネルに固定しているため,すべての個体は平均して良い評価値を得る.
そして,局所解からの脱出方法は偵察バチフェーズのみである.
これに対し提案手法では,カーネル関数が更新される際に,
他のカーネル関数で使っていたパラメータをそのまま使用する.
そのため,カーネル関数が更新される際には偵察バチフェーズのようなランダム性が生じる.
これにより,探索の多様性と局所解からの脱出の面で先行研究よりも優れた手法となった
可能性が考えられる.
また,提案手法では,個体表現はハイパーパラメータの組の5次元だが,
先行研究での個体表現はハイパーパラメータに加え,特徴量を含めた40次元である.
また,提案手法はカーネル関数が変わると探索空間が変化するためさらに低次元の探索となる.
両手法間で探索次元が大きく異なるため,
本研究で行った実験では先行研究は十分に探索できていなかった可能性も考えられる.