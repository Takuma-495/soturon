\section{考察}
まず,表~\ref{result1}より,デフォルトパラメータで最も分類精度の高いカーネル関数はrbfの99.68\%であり,
先行研究と提案手法の分類精度は99.88\%,99.91\%と両手法とも
デフォルトパラメータよりも良い分類精度を得ることができた.
そして,提案手法は分類精度の面では,先行研究よりも0.03\%良い結果が得られた.
一方実行時間の面では提案手法は先行研究よりも約3.7時間劣る結果となった.
また,表~\ref{result2},\ref{result3}より
検知率,誤警報率,適合率,F値のすべてで提案手法のほうが良いという結果になった.
そのため分類精度に加えて,
侵入検知問題に使用するモデルとしても提案手法のほうが良いパラメータを得ている.

提案手法の実行時間が先行研究の実行時間よりも長くなってしまった原因について考察する.
まず,提案手法と先行研究の大きな違いとして,特徴選択の有無が挙げられる.
特徴選択によって,データセットの次元削減が行われるとカーネル関数の計算が低コストになり,
SVMの計算コストが低減する.
また,多くの特徴を削減した個体はモデルの精度が低下する可能性があるが,そのような個体は低次元の計算になるため,
計算時間は短くなる.ゆえに評価値の低い個体に対して多くの計算資源を使う可能性も低くなる.
そのため特徴選択を行ったほうが実行時間は短くなると考えられる.
次に,提案手法ではrbf以外のカーネル関数も扱い,個体によってカーネル関数が異なる点である.
rbfカーネルはハイパーパラメータが1個であるが,シグモイドカーネルは2個,多項式カーネルは3個とrbfよりも多い,
そのためシグモイド,多項式はハイパーパラメータ空間が広く,計算コストの高い個体も生成されやすい.
また,線形カーネルは元の次元で内積計算を行うためrbfカーネルよりも計算コストは高い.
そのため,提案手法の最適化過程では,先行研究よりも評価コストの高い個体を多く評価する必要がある.
表~\ref{result1}のデフォルトパラメータの結果から,rbfと多項式の分類精度が高いことが分かるように.
提案手法の最適化過程ではrbfと多項式の個体がほとんどであった.
そのため多項式カーネルの次元数の範囲をさらに拡張して実験を行えば,より実行時間が長くなると考えられる.

次に,提案手法で最終的に得られた最良のモデルがすべてrbfカーネルを使用していたことについて考察する.

表~\ref{result1}のデフォルトパラメータの結果から,
rbf,多項式,線形,シグモイドの順で分類精度が高かった.
特に,シグモイドは著しく他のカーネル関数よりも分類精度が劣っていた.
提案手法の最適化過程を見るとやはりシグモイドカーネルを使用した評価値の高い個体はごく少数であり,
その評価値はrbfや多項式の個体には及ばなかった.

結局同じカーネル関数だったのになぜ提案手法のほうがよくなったか??
